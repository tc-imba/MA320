\documentclass{article}
\usepackage{enumerate}
\usepackage{amsmath}
\usepackage{amssymb}
\usepackage{graphicx}
\usepackage{subfigure}
\usepackage{geometry}
\usepackage{caption}
\usepackage{indentfirst}
\geometry{left=3.0cm,right=3.0cm,top=3.0cm,bottom=4.0cm}
\usepackage{ctex}
\renewcommand{\thesection}{Ex.}
\DeclareMathOperator*{\suplim}{\overline{\lim}}
\DeclareMathOperator*{\inflim}{\underline{\lim}}

\title{MA320\ 抽象代数\ 作业六}
\author{刘逸灏 515370910207}
\date{\today}

\begin{document}
\maketitle

\section{2.6/1}
$$\phi(12)=2\cdot(2^2-2^1)=4$$
$$Aut(G)=\{\alpha_1,\alpha_5,\alpha_7,\alpha_{11}\}$$
\begin{center}
\begin{tabular}{c|cccc}
$\cdot$ & 1 & 5 & 7 & 11 \\\hline
1 & 1 & 5 & 7 & 11 \\
5 & 5 & 1 & 11 & 7\\
7 & 7 & 11 & 1 & 5\\
11 & 11 & 7 & 5 & 1\\
\end{tabular}
\end{center}

$$K_4=\{(1),(1\ 2)(3\ 4),(1\ 3)(2\ 4),(1\ 4)(2\ 3)\}$$

作一一映射$\phi:Aut(G)\leftrightarrow K_4$使得$\alpha_1\leftrightarrow 1,\alpha_5\leftrightarrow 2,\alpha_7\leftrightarrow 3,\alpha_{11}\leftrightarrow 4$即为一同构映射。

\section{2.6/2}
由于$K_4$同构于上题中的$Aut(G)$,故不是一个循环群,而$Z_4$是一个循环群,易知$K_4$不同构于$Z_4$。

\section{2.6/3}
$$Aut(Z)\cong Z_2$$
$$Aut(Z_3)=\{\alpha_1,\alpha_2\}\cong Z_2$$

故得证。

\section{2.6/4}
\begin{enumerate}[(i)]
\item
$$o(g^s)=\frac{o(g)}{(s,o(g))}=\frac{t}{(s,t)}=t$$
由于$g$与$g^s$有相同的阶,且$g^s\in\langle g\rangle$,故显然$\langle g\rangle=\langle g^s\rangle$。
\item
$$o(g^s)=\frac{o(g)}{(s,o(g))}=\frac{t}{(s,t)}=\frac{t}{k}$$
\end{enumerate}

\section{2.6/5}
\begin{enumerate}[(i)]
\item
$$g_1^{t_1}=g_2^{t_2}=e$$
$$(g_1g_2)^{[t_1,t_2]}=e$$
现证明$t=[t_1,t_2]$,假设$t'<t$,则根据最小公倍数定义可知不存在$t_1\mid t'$且$t_2\mid t'$,故得证。
\item
设$o(g_1)=t_1',o(g_2)=t_2'$,且$(t_1',t_2)=(t_1,t_2')=1$
$$o(g_1g_2)=t_1t_2$$
$$(g_1g_2)^{t_1t_2}=(g_1^{t_2})^{t_1}(g_2^{t_1})^{t_2}=e$$
$$o(g_1^{t_2})=\frac{t_1'}{(t_1',t_2)}=t_1'\Longrightarrow t_1'\mid t_1$$
$$o(g_2^{t_1})=\frac{t_2'}{(t_2',t_1)}=t_2'\Longrightarrow t_2'\mid t_2$$

由于$(t_1',t_2)=(t_1,t_2')=(t_1,t_2)=1$,易得$t_1=t_1',t_2=t_2'$,得证。

\end{enumerate}

\section{2.6/6}
$\forall s$使得$a^s\in\langle a^m\rangle\cap\langle a^n\rangle$,则$m\mid s$且$n\mid s$,故$[m,n]\mid s$,由$s$的任意性得证。

\end{document}
 
