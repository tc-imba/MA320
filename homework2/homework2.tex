\documentclass{article}
\usepackage{enumerate}
\usepackage{amsmath}
\usepackage{amssymb}
\usepackage{graphicx}
\usepackage{subfigure}
\usepackage{geometry}
\usepackage{caption}
\usepackage{indentfirst}
\geometry{left=3.0cm,right=3.0cm,top=3.0cm,bottom=4.0cm}
\usepackage{ctex}
\renewcommand{\thesection}{Ex.}
\DeclareMathOperator*{\suplim}{\overline{\lim}}
\DeclareMathOperator*{\inflim}{\underline{\lim}}

\title{MA320\ 抽象代数\ 作业二}
\author{刘逸灏 515370910207}
\date{\today}

\begin{document}
\maketitle

\section{2.1/6}
若$x^2\neq e$,则$(x^{-1})^2\neq e$,且$a\neq a^{-1}$。设$S=\varnothing$,只需每次从有限群$G$中取出一个元$x$,若$x^2\neq e,x\not\in S$,则将$x$和$x^{-1}$都加入$S$中,所以$S$中必定有偶数个元素。当$G$中所有元都被取出时,其中满足$x^2\neq e$的元必定都在$S$中,故有偶数个这样的元。

\section{2.1/7}
由上题得,$x^2\neq e$的元有偶数个,故在偶数阶群中总共有偶数个元,故满足$x^2=e$的元也有偶数个。

\section{2.1/8}
$\forall a,b\in G$,$a^2=e,b^2=e$,则
$$(ab)^2=abab=e=a^2b^2$$
$$ba=ab$$

故$G$为Abel群。

\section{2.2/4}
$\forall g,h\in C_G(A)$,$\forall a\in A$,则
$$(gh)a=g(ha)=g(ah)=(ga)h=(ag)h=a(gh)$$

故乘法封闭,$C_G(A)$为$G$的一个子群,但$A$中元素不一定都满足交换律,故$C_G(A)$不一定包含$A$。

\section{2.2/5}


\section{2.2/7}

\end{document}
 
