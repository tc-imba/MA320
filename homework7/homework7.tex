\documentclass{article}
\usepackage{enumerate}
\usepackage{amsmath}
\usepackage{amssymb}
\usepackage{graphicx}
\usepackage{subfigure}
\usepackage{geometry}
\usepackage{caption}
\usepackage{indentfirst}
\geometry{left=3.0cm,right=3.0cm,top=3.0cm,bottom=4.0cm}
\usepackage{ctex}
\renewcommand{\thesection}{Ex.}
\DeclareMathOperator*{\suplim}{\overline{\lim}}
\DeclareMathOperator*{\inflim}{\underline{\lim}}

\title{MA320\ 抽象代数\ 作业七}
\author{刘逸灏 515370910207}
\date{\today}

\begin{document}
\maketitle

\section{2.8/1}
充分性:若群$G$是合数阶群,且阶数$n$可分解为至少两个素因数的幂$p_1^{r_1}$和$p_2^{r_2}$,则根据Sylow定理可知$G$中存在两个阶不同且都大于1的子群,与条件矛盾,故$G$一定是素数阶群。又由于$G$是单群,必须由一个生成元生成,故是循环群。

必要性:根据Lagrange定理可知,对于素数阶循环Abel群,其子群的阶只能为$p$或$1$,易知$G$为单群。

\section{2.8/2}
设$g\not\in H$,则$gH$是$H$在$G$中的一个左陪集,且$H$也是$G$的一个左陪集。又由于$g\not\in H$可知$H\cap gH=\varnothing$,且$G$中只有两个$H$的左陪集,故$G=H\cup gH$。同理可得$G=H\cup Hg$,则易知$gH=Hg$,即$H$是$G$的正规子群。

\section{2.8/3}
$$a(ba^{-1}b^{-1})\in M$$
$$(aba^{-1})b^{-1}\in N$$
$$a(ba^{-1}b^{-1})=(aba^{-1})b^{-1}\in M\cap N=\{e\}$$
$$aba^{-1}b^{-1}=e$$
$$ab=ba$$

\section{2.8/4}
根据Sylow定理可知有Sylow-p子群$H$的阶为$p^s$,其中$p^s\mid n$且$p^{s+1}\nmid n$,对于$\forall h\in H,h\neq e_H$,根据Lagrange定理可知$h$的阶为$p^t,1\leqslant t\leqslant s$,故$h^{p^{t-1}}$的阶为$p$,得证。

\section{2.8/5}
作映射$\phi:x^{-1}Hx\to xN_G(H)$,则
$$u^{-1}Hu=v^{-1}Hv\to uN_G(H)=vN_G(H)u^{-1}Hu=v^{-1}Hv\to uN_G(H)=vN_G(H)$$

故$\phi$是一个一一映射,得证。

\section{2.8/9}
根据Lagrange定理可知$C(G)$的阶只能为$p,p^2$。当阶为$p^2$时,显然有$C(G)=G$,故$G$为Abel群。当阶为$p$时,$G/C(G)$的阶为$p$是循环群,故$G$为Abel群。

\section{2.8/10}
根据Lagrange定理可知$C(G)$的阶只能为$p,p^2,p^3$。显然当阶为$p^3$时$G$为Abel群,与条件矛盾。当阶为$p^2$时,$G/C(G)$的阶为$p$是循环群,故$G$为Abel群,与条件矛盾。故$C(G)$的阶只能为$p$,故其同构于$Z_p$。

\end{document}
 
