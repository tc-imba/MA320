\documentclass{article}
\usepackage{enumerate}
\usepackage{amsmath}
\usepackage{amssymb}
\usepackage{graphicx}
\usepackage{subfigure}
\usepackage{geometry}
\usepackage{caption}
\usepackage{indentfirst}
\geometry{left=3.0cm,right=3.0cm,top=3.0cm,bottom=4.0cm}
\usepackage{ctex}
\renewcommand{\thesection}{Ex.}
\DeclareMathOperator*{\suplim}{\overline{\lim}}
\DeclareMathOperator*{\inflim}{\underline{\lim}}

\title{MA320\ 抽象代数\ 作业四}
\author{刘逸灏 515370910207}
\date{\today}

\begin{document}
\maketitle

\section{2.4/1}
没有,根据拉格朗日定理,子群的阶数必定为群阶数的因子,而这里6不是20的因子。

\section{2.4/2}
充分性:当等价关系$\mathop{\sim}\limits^H$和$(\mathop{\sim}\limits^H)'$相等时,对于$a\in G$,可找出所有由这两个相等等价关系的元素$X$组成的集合,且$x\in aH,x\in Ha$,可推出左陪集$aH$和右陪集$Ha$也相等,故满足$H$是$G$的正规子群。

必要性:当$H$是$G$的正规子群时,对于$a\in G$,可知$aH=Ha$,现取$x\in aH$,则$x$为与$a$有等价关系$\mathop{\sim}\limits^H$的元。取$x\in Ha$,则$x$为与$a$有等价关系$(\mathop{\sim}\limits^H)'$,故这连个等价关系相等。

\section{2.4/3}
$\forall h\in H$, $\exists ab,ahb\in aH\cdot bH$,$(ahb)H=(ab)H$,即$(ab)^{-1}(ahb)=b^{-1}hb\in H$。由于$b$和$h$的任意性可知$bH=Hb$,即$H$为$G$的正规子群。

\section{2.4/5}
$$S_3=\{(1),(1\ 2),(1\ 3),(2\ 3),(1\ 2\ 3),(1\ 3\ 2)\}$$
$$K_4=\{(1),(1\ 2)(3\ 4),(1\ 3)(2\ 4),(1\ 4)(2\ 3)\}$$
$$(1)K_4=\{(1),(1\ 2)(3\ 4),(1\ 3)(2\ 4),(1\ 4)(2\ 3)\}$$
$$(1\ 2)K_4=\{(1\ 2),(3\ 4),(1\ 3\ 2\ 4),(1\ 4\ 2\ 3)\}$$
$$(1\ 3)K_4=\{(1\ 3),(1\ 2\ 3\ 4),(2\ 4),(1\ 4\ 3\ 2)\}$$
$$(2\ 3)K_4=\{(2\ 3),(1\ 3\ 4\ 2),(1\ 2\ 4\ 3),(1\ 4)\}$$
$$(1\ 2\ 3)K_4=\{(1\ 2\ 3),(1\ 3\ 4),(2\ 4\ 3),(1\ 4\ 2)\}$$
$$(1\ 3\ 2)K_4=\{(1\ 3\ 2),(2\ 3\ 4),(1\ 2\ 4),(1\ 4\ 3)\}$$
$$K_4(1)=\{(1),(1\ 2)(3\ 4),(1\ 3)(2\ 4),(1\ 4)(2\ 3)\}$$
$$K_4(1\ 2)=\{(1\ 2),(3\ 4),(1\ 4\ 2\ 3),(1\ 3\ 2\ 4)\}$$
$$K_4(1\ 3)=\{(1\ 3),(1\ 4\ 3\ 2),(2\ 4),(1\ 2\ 3\ 4)\}$$
$$K_4(2\ 3)=\{(2\ 3),(1\ 2\ 4\ 3),(1\ 3\ 4\ 2),(1\ 4)\}$$
$$K_4(1\ 2\ 3)=\{(1\ 2\ 3),(2\ 4\ 3),(1\ 4\ 2),(1\ 3\ 4)\}$$
$$K_4(1\ 3\ 2)=\{(1\ 3\ 2),(1\ 4\ 3),(2\ 3\ 4),(1\ 2\ 4)\}$$

由以上计算可得$K_4$是$S_4$的正规子群,且$S_3$为一个左陪集。

\section{2.4/6}
由$H\cdot K$的定义知$H\cdots K$是所有形如$hK$的左陪集的并,因不同的左陪集的交为空集,故其等于群$H$的子群$H\cap K$在$H$中的左陪集个数。$\forall t\in H\cap K$,$hk=(ht)(t^{-1}k)$,$ht\in H, t^{-1}k\in H\cap K$,故群$H$的子群$H\cap K$在$H$中的左陪集个数和形如$hK$的左陪集个数相等,为$|H||K|/|H\cap K|$。

\end{document}
 
