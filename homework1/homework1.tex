\documentclass{article}
\usepackage{enumerate}
\usepackage{amsmath}
\usepackage{amssymb}
\usepackage{graphicx}
\usepackage{subfigure}
\usepackage{geometry}
\usepackage{caption}
\usepackage{indentfirst}
\geometry{left=3.0cm,right=3.0cm,top=3.0cm,bottom=4.0cm}
\usepackage{ctex}
\renewcommand{\thesection}{Ex.}
\DeclareMathOperator*{\suplim}{\overline{\lim}}
\DeclareMathOperator*{\inflim}{\underline{\lim}}

\title{MA320\ 抽象代数\ 作业一}
\author{刘逸灏 515370910207}
\date{\today}

\begin{document}
\maketitle

\section{1/1}
假设有限集有$n$个元素,且由$f$为$A$的变换可知$f(A)\subseteq A$。

充分性:已知$f$为单射,则$f(A)$也有$n$个元素。由于$f(A)$和$A$都有$n$个元素,易知$f(A)=A$。假设$f$不为满射,即$\exists y\in f(A),f^{-1}(y)\not\in A$,显然这与$f(A)=A$矛盾,故$f$为满射。

必要性:已知$f$为满射,则$\forall y\in f(A)$,$\exists x\in A$使得$f(x)=y$,易知$f(A)$中至少有$n$个元素。假设$f$不为单射,则$f(A)$小于$n$个元素,矛盾,故$f$为单射。

\section{1/2}
已知$f$为$n$维线性空间$V$的线性变换,根据线性空间的性质可知
$$\dim Ker(f)+\dim Im(f)=n$$

若$f$为单射,则$Ker(f)=\{\mathbf{0}\}$,$\dim Im(f)=n$,$Im(f)=V$,可推出$f$为满射。以上推理均可逆,故必要性也得证。

\section{1/3}
$A$到$B$有$n^m$个映射,有$C_m^n$个单射,$A$有$m^{(m^2)}$个二元运算。

\section{1/5}
根据高等代数关于实对称矩阵合同的定理,$S$中每个矩阵均合同于一个实对角矩阵,通过矩阵的初等变换,其等价于
$$
M=\begin{pmatrix}
I_p & 0 & 0 \\
0 & -I_q & 0 \\
0 & 0 & 0
\end{pmatrix},p+q\leqslant n
$$

令
$$M_i=
\begin{pmatrix}
I_p & 0 & 0 \\
0 & -I_q & 0 \\
0 & 0 & 0
\end{pmatrix},p+q=i
$$

则$\{M_i\}$为$M$的一个划分,相应$S$的商集为$\{r(S)=i\}$。

\section{1/6}
根据高等代数关于实对称矩阵相似的定理,$S$中每个矩阵均合同于一个实对角矩阵,其等价于
$$
M=\begin{pmatrix}
\lambda_1 & 0 & \cdots & 0 \\
0 & \lambda_2 & \cdots & 0 \\
\vdots & \vdots & \ddots & \vdots \\
0 & 0 & \cdots & \lambda_n
\end{pmatrix},\lambda_1\leqslant\lambda_2\leqslant\cdots\leqslant\lambda_n
$$

令
$$
M_i=\begin{pmatrix}
\lambda_i & 0 & \cdots & 0 \\
0 & \lambda_i & \cdots & 0 \\
\vdots & \vdots & \ddots & \vdots \\
0 & 0 & \cdots & \lambda_n
\end{pmatrix},\lambda_i<\lambda_{i+1}\leqslant\cdots\leqslant\lambda_n
$$

则$\{M_i\}$为$M$的一个划分,相应$S$的商集为$\{r(S)=i\}$。

\section{1/7}
根据Bézout's identity,$a,b\in N^*$互质的充要条件是存在整数$x,y$使得$ax+by=1$。这里,令$a=\bar{i},b=n,x=\bar{j}$,则可推得
$$\overline{ij}+ny=1$$

由于$n|ny$,可得$\overline{ij}=\bar{1}$,根据该充要关系可推得$\bar{i}$与$n$互质时满足性质。

\section{2.1/1}
易知两个下三角矩阵的乘积仍为下三角矩阵,两个主对角矩阵的乘积仍为主对角矩阵,故矩阵乘法是$N,D$的运算。由矩阵乘法运算性质可知其满足结合律,设单位元为$I$,显然$NI=N,IN=N,DI=D,ID=D$。由于$N$为非奇异矩阵,$N$对矩阵乘法可逆,只需证明$N^{-1}$仍为下三角矩阵。设
$$
N=\begin{pmatrix}
a_{11} & 0 & \cdots & 0 \\
a_{21} & a_{22} & \cdots & 0 \\
\vdots & \vdots & \ddots & \vdots \\
a_{n1} & a_{n2} & \cdots & a_{nn}
\end{pmatrix}
$$
$$
N^{-1}=\frac{N^*}{|N|}=\frac{1}{|N|}\begin{pmatrix}
N_{11} & N_{12} & \cdots & N_{1n} \\
N_{21} & N_{22} & \cdots & N_{2n} \\
\vdots & \vdots & \ddots & \vdots \\
N_{n1} & N_{n2} & \cdots & N_{nn}
\end{pmatrix}
$$

由于$N$为下三角矩阵,可得$N_{ij}=0,i>j$,所以$N^{-1}$也为下三角矩阵,所以$N$对矩阵乘法作成群。

同时,根据对角矩阵的性质,若对角线上都非零,则其可逆,且逆矩阵对角线上每个值都为原来的倒数,故$D$也对矩阵乘法作成群。

\section{2.1/2}
令$(a,b),(c,d),(e,f)\in G,e=(1,0),(a,b)^{-1}=(1/a,-b/a)$
$$[(a,b)(c,d)](e,f)=(ac,ad+b)(e,f)=(ace,acf+ad+b)=(a,b)(ce,cf+d)=(a,b)[(c,d)(e,f)]$$
$$e(a,b)=(a,b)e=(a,b)$$
$$(a,b)(a,b)^{-1}=e$$

故$G$对此二元运算作成群。

\section{2.1/3}
令$f,g,h\in G^\Omega$,$e$为等于$G_e$的恒等映射,$f^{-1}(a)=[f(a)]^{-1}$
$$[(fg)h](a)=[f(a)g(a)]h(a)=f(a)g(a)h(a)=f(a)[g(a)h(a)]=[f(gh)](a)$$
$$(ef)(a)=(fe)(a)=f(a)$$
$$f(a)f^{-1}(a)=G_e=e$$

故$G^\Omega$是群。

\section{2.1/4}
对$G$的任一元$g$,用$g_r$表示$g$的右逆元,$g_r$的右逆元记为$(g_r)_r$,于是
$$g_r(g_r)_r=e_r=gg_r$$
$$g_rg=g_r(ge_r)=g_rg(g_r(g_r)_r)=g_r(gg_r)(g_r)_r=g_re_r(g_r)_r=g_r(g_r)_r=e_r$$
$$e_rg=(gg_r)g=g(g_rg)=ge_r=g$$

这表明$e_r$是$G$的单位元,即$G$是群。

\section{2.1/6}
若$x^2\neq e$,则$(x^{-1})^2\neq e$,且$a\neq a^{-1}$。设$S=\varnothing$,只需每次从有限群$G$中取出一个元$x$,若$x^2\neq e,x\not\in S$,则将$x$和$x^{-1}$都加入$S$中,所以$S$中必定有偶数个元素。当$G$中所有元都被取出时,其中满足$x^2\neq e$的元必定都在$S$中,故有偶数个这样的元。

\end{document}
 
