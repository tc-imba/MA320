\documentclass{article}
\usepackage{enumerate}
\usepackage{amsmath}
\usepackage{amssymb}
\usepackage{graphicx}
\usepackage{subfigure}
\usepackage{geometry}
\usepackage{caption}
\usepackage{indentfirst}
\geometry{left=3.0cm,right=3.0cm,top=3.0cm,bottom=4.0cm}
\usepackage{ctex}
\renewcommand{\thesection}{Ex.}
\DeclareMathOperator*{\suplim}{\overline{\lim}}
\DeclareMathOperator*{\inflim}{\underline{\lim}}

\title{MA320\ 抽象代数\ 作业五}
\author{刘逸灏 515370910207}
\date{\today}

\begin{document}
\maketitle

\section{2.5/1}
设$f$的阶为$m$,$f(g)$的阶为$n$,即 
$$g^m=e_G$$
$$f(g)^n=e_H$$
$$f(g^m)=f(g)^m=f(e_G)=e_H$$

若$n\nmid m$,则$f(g)^m=f(g)^{m\mod n}\neq e_H$,矛盾,故$f(g)$的阶整除$g$的阶。

\section{2.5/2}
$$f(KM)=f(K)f(M)=e_Hf(M)=f(M)$$
$$f^{-1}(f(M))=KM$$

\section{2.5/3}
根据定理5.8,已知$N$是$G$的正规子群,$M$是$G$的子群,故$N\cap M$为$M$的正规子群,且$N\leqslant M\to N\cap M=N$,即$N\lhd M$。

\section{2.5/4}
考虑行列式映射$GL_n(R)\to R^*:\phi(x)\to\det(x)$,由行列式运算性质可知这是一个群满同态,且$\ker\phi=SL_n(R)$。根据定理5.6(i)可知,$GL_n(R)/SL_n(R)\cong R^*$。

\section{2.5/5}
设$H=C(G)$,$G/H$为一个循环群,其生成元为$aH$,令$x,y \in G$,则$\exists m,n$使得$xH=(aH)^m=a^mH,yH=(aH)^n=a^nH$,$\exists h_1,h_2\in H$使得$x=a^mh_1,y=a^nh_2$,易知
$$xy=(a^mh_1)(a^nh_2)=a^{m+n}h_1h_2=(a^nh_2)(a^mh_1)=yx$$

故$G$是Abel群。

\end{document}
 
