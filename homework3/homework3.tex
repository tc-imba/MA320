\documentclass{article}
\usepackage{enumerate}
\usepackage{amsmath}
\usepackage{amssymb}
\usepackage{graphicx}
\usepackage{subfigure}
\usepackage{geometry}
\usepackage{caption}
\usepackage{indentfirst}
\geometry{left=3.0cm,right=3.0cm,top=3.0cm,bottom=4.0cm}
\usepackage{ctex}
\renewcommand{\thesection}{Ex.}
\DeclareMathOperator*{\suplim}{\overline{\lim}}
\DeclareMathOperator*{\inflim}{\underline{\lim}}

\title{MA320\ 抽象代数\ 作业三}
\author{刘逸灏 515370910207}
\date{\today}

\begin{document}
\maketitle

\section{2.3/1}
$$1\to3\to5\to2$$
$$2\to1\to1\to1$$
$$3\to2\to3\to3$$
$$4\to4\to4\to5$$
$$5\to5\to2\to4$$
$$(4\ 5\ 2)(5\ 2\ 3)(3\ 2\ 1)=(1\ 2)(4\ 5)$$

\section{2.3/2}
$$(i_1\ i_2\cdots i_t)=(i_1\ i_t)\cdots(i_1\ i_4)(i_1\ i_3)(i_1\ i_2)$$

即$t$-轮换可表示为$t$个对换的积。由于偶数个对换的乘积是偶置换,奇数个对换的乘积是奇置换,故得证。

\section{2.3/3}
设$\sigma=(i_1\ i_2\cdots i_t)=(i_1\ i_t)\cdots(i_1\ i_4)(i_1\ i_3)(i_1\ i_2)$,$\sigma^n$即为对于每一个元素进行轮换,
$$i_1\to i_2\to i_3\to\cdots\to i_1$$
$$i_2\to i_3\to i_4\to\cdots\to i_2$$
$$\cdots$$
$$i_t\to i_1\to i_4\to\cdots\to i_t$$

易知对于每个元素,进行$t$次轮换可得到自身,即$\sigma^t=e$,故$t$-轮换的阶是$t$。

\section{2.3/5}
考虑置换
$$\sigma=(1\ 2)(2\ 3)\cdots(i\ i+1)\cdots(n-1\ n)$$
$$1\to1\to1\to\cdots\to1\to1\to2$$
$$2\to2\to2\to\cdots\to2\to3\to3$$
$$\cdots$$
$$n-1\to n\to n\cdots\to n\to n\to n$$
$$n\to n-1\to n-2\cdots\to 3\to 2\to 1$$

即$\sigma=(1\ 2\ 3\cdots n)$,故题设为$S_n$的一个生成元集

\section{2.3/6}
当$n$为偶数,$$\sigma=(1\ n)(2\ n-1)\cdots(\frac{n-2}{2}\ \frac{n+2}{2})$$

当$n$为奇数,$$\sigma=(1\ n)(2\ n-1)\cdots(\frac{n-1}{2}\ \frac{n+1}{2})$$

故当$n=4k$或$n=4k+3$时,$\sigma$为奇数个对换的乘积,为奇置换。当$n=4k+1$或$n=4k+2$时,$\sigma$为偶数个对换的乘积,为偶置换。($k\in N$)

\end{document}
 
